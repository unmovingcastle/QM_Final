\documentclass[12pt]{report}
% \documentclass[12pt,letterpaper]{article}
\usepackage{preamble}

% \newcommand\course{}
% \newcommand\hwnumber{}
% \newcommand\userID{Jason Yao}
% \newcommand\collaborator{V. Houh, G. Summermatter}
\newcommand{\sectionbreak}{\clearpage}
\newcommand*\mean[1]{\overline{#1}}
\newcommand\np{\vspace{0.5cm}\\} % as in "next point"
\newcommand\nl{\vspace{5mm}\hrule\vspace{5mm}\noindent} % as in "new line"

\title{The Missing Steps in Griffiths}
\author{Jason Yao}
\date{June 9, 2021}


\graphicspath{{Images/}}
\setcounter{tocdepth}{3}

\begin{document}

\maketitle

\numberwithin{equation}{section}

\tableofcontents

\clearpage
\begin{multicols}{2}
In this document, I attempt to include all the explanations to things that confused the heck out of me during my first read through (and the second, third, you get the idea). Usually, these are derivations that have around 10,000 missing steps in between --- the kind of derivations that Griffiths seems to enjoy making. Occasionally, I will include alternative ways to interpret/derive results. These mainly come from Brant Carlson's YouTube Channel. Finally, please understand that this document is by no means intended to be a self-contained piece --- for that we have these things that are commonly known to scholars as ``books'' and to publishers as ``books (799 USD)'' ---instead, this document is created to assist you in reading the textbook. When referring to previous equations, instead of something like ``see Eq. (420)'' (and then you have to scroll through two-hundred pages on a computer), I will try as much as possible to utilize mouse-over tooltips, like \tooltip{so}{
\parbox{10cm}
{ The end-all-be-all equation: 
\begin{equation*}
    i\hbar\pdv{\Psi}{t}=-\frac{\hbar^2}{2m}\pdv[2]{\Psi}{x}+V\Psi
\end{equation*}
}}. 
The tooltip feature, however, is pretty much only supported by Adobe software, so you will most likely need to read this using their (free) Acrobat Reader. (If you truly are a rebel who hates being told what to do, feel free to use Preview on you Mac, and good luck with that.) Alternatively, you can simply go print this out using Chem Department's printer, and on your way back to the Physics Lounge, see if you can \st{steal} borrow some staplers and tapes, among other things, for the greater good of mankind.
\end{multicols}


\chapter{}

\setcounter{section}{3}
\section{Normalization}

\textit{The Schrödinger equation has the remarkable property that it automatically preserves the normalization of the wave function}. Suppose the wave function is normalized at time $t=0$, it will stay normalized. Suppose we already have a normalized wave function $\Psi(x,t)$ so that 
\begin{equation*}
    \int_{-\infty}^{\infty}\abs{\Psi(x,t)}^2\dd x = 1
\end{equation*}
Saying that the above integral will stay normalized is equivalent to saying that the time derivative of the \tooltip{LHS}{Left Hand Side} is zero (i.e. constant in time). So we are going to prove the following relation
\begin{equation}
    \dv[]{}{t}\int_{-\infty}^{\infty}\abs{\Psi(x,t)}^2\dd x = 0
    \label{eq:1_4_1}
\end{equation}
To begin with,
\begin{equation*}
    \dv[]{}{t}\int_{-\infty}^{\infty}\abs{\Psi(x,t)}^2\dd x = \int_{-\infty}^{\infty}\pdv{}{t}\abs{\Psi(x,t)}^2\dd x 
\end{equation*}
By the product rule,
\begin{equation*}
    \pdv{}{t}\abs{\Psi}^2 = \pdv{t}(\Psi^*\Psi) = \Psi^*\pdv{\Psi}{t}+\pdv{\Psi^*}{t}\Psi
\end{equation*}
From the \tooltip{Schrödinger equation}
{
\parbox{10cm}{
\begin{equation*}
    \setlength{\jot}{13pt}
    \begin{aligned}
            i\hbar\pdv{\Psi}{t} &= -\frac{\hbar^2}{2m}\pdv[2]{\Psi}{x}+V\Psi \\
    \implies-\hbar \pdv{\Psi}{t} &= \frac{-i\hbar^2}{2m}\pdv[2]{\Psi}{x}+Vi\Psi\\
    \implies-\pdv{\Psi}{t} &= \frac{-i\hbar}{2m}\pdv[2]{\Psi}{x}+\frac{Vi}{\hbar}\Psi\\
    \end{aligned}
\end{equation*}
}
}, we can derive the following
\begin{equation}
    \pdv{\Psi}{t} = \frac{i\hbar}{2m}\pdv[2]{\Psi}{x}-\frac{i}{\hbar}V\Psi
    \label{eq:1_4_2}
\end{equation}
And the complex conjugate of the above equation is
\begin{equation}
    \pdv{\Psi^*}{t} = -\frac{i\hbar}{2m}\pdv[2]{\Psi^*}{x}+\frac{i}{\hbar}V\Psi^*
    \label{eq:1_4_3}
\end{equation}
Thus, finishing the product rule,
\begin{equation*}
\setlength{\jot}{13pt}
\begin{aligned}
        \pdv{}{t}\abs{\Psi}^2 &= \Psi^*\pdv{\Psi}{t}+\pdv{\Psi^*}{t}\Psi = \Psi^*\cdot\qty(\frac{i\hbar}{2m}\pdv[2]{\Psi}{x}-\frac{i}{\hbar}V\Psi)+\Psi\qty(-\frac{i\hbar}{2m}\pdv[2]{\Psi^*}{x}+\frac{i}{\hbar}V\Psi^*) \\
        &= \frac{i\hbar}{2m}\qty(\Psi^*\pdv[2]{\Psi}{x}-\pdv[2]{\Psi^*}{x}\Psi) = \pdv{}{x}\qty[\frac{i\hbar}{2m}\qty(\Psi^*\pdv[]{\Psi}{x}-\pdv[]{\Psi^*}{x}\Psi)]
\end{aligned} 
\end{equation*}
If one so desires, the last step above can be easily \tooltip{confirmed}{
\parbox{10cm}{
\begin{align*}
    &\pdv{}{x}\qty[\frac{i\hbar}{2m}\qty(\Psi^*\pdv[]{\Psi}{x}-\pdv[]{\Psi^*}{x}\Psi)] 
    \\&= \frac{i\hbar}{2m}\qty[\qty(\pdv{\Psi^*}{x}\pdv{\Psi}{x}+\Psi^*\pdv[2]{\Psi}{x})-\qty(\pdv[2]{\Psi^*}{x}\Psi+\pdv{\Psi^*}{x}\pdv{\Psi}{x})]
    \\&=\frac{i\hbar}{2m}\qty(\Psi^*\pdv[2]{\Psi}{x}-\pdv[2]{\Psi^*}{x}\Psi)
\end{align*}
}
} by applying the product rule, i.e. going from RHS to LHS.
Now that we have established the equation 
\begin{equation}
    \pdv{}{t}\abs{\Psi}^2  = \pdv{}{x}\qty[\frac{i\hbar}{2m}\qty(\Psi^*\pdv[]{\Psi}{x}-\pdv[]{\Psi^*}{x}\Psi)],
    \label{eq:1_4_4}
\end{equation}
the integral in Equation \ref{eq:1_4_1} can be evaluated
\begin{align*}
    &\dv{t}\int_{-\infty}^{\infty}\abs{\Psi(x,t)}^2\dd x\\
    &= 
    \int_{-\infty}^{\infty}\pdv{t}\abs{\Psi(x,t)}^2\dd x 
    \\&=
    \int_{-\infty}^{\infty}\pdv{x}\qty[\frac{i\hbar}{2m}\qty(\Psi^*\pdv[]{\Psi}{x}-\pdv[]{\Psi^*}{x}\Psi)]\dd x\\
    &= \frac{i\hbar}{2m}\qty(\Psi^*\pdv[]{\Psi}{x}-\pdv[]{\Psi^*}{x}\Psi)\eval_{-\infty}^{\infty}\\
    &=0
\end{align*}
The last equation above holds because the wave function $\Psi(x,t)$ must go to zero as $x$ approaches $\pm$ infinity for the wave function to be normalizable. As such, we have shown that the time derivative of the integral is constant, i.e. the normalization will not be affected by time. 


\section{Momentum}

\subsection{velocity and momentum}
We derive the expressions for $\expval{v}$ as well as $\expval{p}$ in this section. From Eq. \tooltip{\ref{eq:1_4_4}}{\parbox{10cm}{
\begin{equation*}
    \pdv{}{t}\abs{\Psi}^2  = \pdv{}{x}\qty[\frac{i\hbar}{2m}\qty(\Psi^*\pdv[]{\Psi}{x}-\pdv[]{\Psi^*}{x}\Psi)]
\end{equation*}}} as well as the fact that 
\begin{equation*}
    \expval{x}=\int_{-\infty}^{\infty}x\abs{\Psi(x,t)}^2\dd x,
\end{equation*} 
it follows that 
\begin{align*}
    \dv{\expval{x}}{t}=\dv{t}\int_{-\infty}^{\infty}x\abs{\Psi(x,t)}^2\dd x = \int_{-\infty}^{\infty}\pdv{t}\qty(x\abs{\Psi(x,t)}^2)\dd x =\int_{-\infty}^{\infty}x\pdv{t}\abs{\Psi(x,t)}^2\dd x
\end{align*}
($x$ can be taken out of the partial derivative, since we are differentiating with respect to $t$), substituting in the expression:
\begin{align*}
    \int_{-\infty}^\infty x\pdv{t}\abs{\Psi(x,t)}^2\dd x &= \int_{-\infty}^\infty x\pdv{}{x}\qty[\frac{i\hbar}{2m}\qty(\Psi^*\pdv[]{\Psi}{x}-\pdv[]{\Psi^*}{x}\Psi)]\dd x\\
    &= \frac{i\hbar}{2m}\int_{-\infty}^\infty x\pdv{}{x}\qty(\Psi^*\pdv[]{\Psi}{x}-\pdv[]{\Psi^*}{x}\Psi)\dd x
\end{align*}
Integration-by-parts leads to
\begin{align*}
    -\frac{i\hbar}{2m}\int_{-\infty}^\infty \dv{x}{x}\qty(\Psi^*\pdv[]{\Psi}{x}-\pdv[]{\Psi^*}{x}\Psi)\dd x + x\qty(\Psi^*\pdv[]{\Psi}{x}-\pdv[]{\Psi^*}{x}\Psi)\eval_{-\infty}^{\infty}
\end{align*}
where the last term on the right, again, goes to zero due to the wave function's being normalizable. We are now left with just the integral on the left. Did not like integration-by-parts? Guess what, we are going to do it again!
\begin{align*}
    -\frac{i\hbar}{2m}\int_{-\infty}^\infty \dv{x}{x}\qty(\Psi^*\pdv[]{\Psi}{x}-\pdv[]{\Psi^*}{x}\Psi)\dd x 
    &= -\frac{i\hbar}{2m}\int_{-\infty}^\infty \Psi^* \pdv{\Psi}{x}-\pdv{\Psi^*}{x}\Psi \dd x \\
    &= \frac{i\hbar}{2m}\qty(\int_{-\infty}^\infty -\Psi^* \pdv{\Psi}{x}\dd x+\int_{-\infty}^\infty \Psi \pdv{\Psi^*}{x}\dd x) \\
     &= \frac{i\hbar}{2m}\qty(\int_{-\infty}^\infty -\Psi^* \pdv{\Psi}{x}\dd x+\int_{-\infty}^\infty -\Psi^* \pdv{\Psi}{x}\dd x+\Psi\Psi^*\eval_{-\infty}^{\infty}) \\
     &= \frac{i\hbar}{2m}\qty(\int_{-\infty}^\infty -2\Psi^* \pdv{\Psi}{x}\dd x)\\
     &= \frac{-i\hbar}{m}\qty(\int_{-\infty}^\infty \Psi^* \pdv{\Psi}{x}\dd x)
\end{align*}
One last time, $\Psi(x,t)$ goes to zero as $x$ approaches $\pm$ infinity. Hence, we have arrived at what is labelled as Equation [31] in my edition of the book.
\begin{equation}
    \dv{\expval{x}}{t}=-\frac{i\hbar}{m}\int\Psi^*\pdv{\Psi}{x}\dd x
\end{equation}
That is,
\begin{equation}
    \expval{v}=\dv[]{\expval{x}}{t}=-\frac{i\hbar}{m}\int\Psi^*\pdv{\Psi}{x}\dd x
\end{equation}
And you know what, I am starting to understand why Griffiths does not write out every last step.
It is a well known fact that the momentum is $p = mv$. It is only natural that
\begin{align}
    \expval{p} = m\expval{v} = -i\hbar\int\Psi^*\pdv{\Psi}{x}\dd x
\end{align}

\subsection{Ehrenfest's theorem (problem 7)}

\chapter{}
% \input{chapter02_01}

\setcounter{section}{1}
\section{The Infinite Square Well}
\subsection{double summation}
If the double summation near the end of this section (the proof that $\sum\abs{c_n}^2=1$) makes you uncomfortable, here is a less compact explanation.
\begin{align*}
    1 = &\int\abs{\Psi(x,0)}^2\dd x = \int\qty(\sum_{m=1}^\infty c_m\psi_m(x))^*\qty(\sum_{n=1}^\infty c_n\psi_n(x))\dd x \\
    =&\int(c_1^*\psi_1^*+c_2^*\psi_2^*+c_3^*\psi_3^*+...)(c_1\psi_1+c_2\psi_2+c_3\psi_3+...) \dd x\\
    =&~c_1^*c_1\int\psi_1^*\psi_1\dd x + c_1^*c_2\int\psi_1^*\psi_2 \dd x+ c_1^*c_3\int\psi_1^*\psi_3 \dd x + ... \\
    &~c_2^*c_1\int\psi_2^*\psi_1\dd x+c_2^*c_2\int\psi_2^*\psi_2\dd x+c_2^*c_3\int\psi_2^*\psi_3\dd x+... \\
    &~c_3^*c_1\int\psi_3^*\psi_1\dd x+c_3^*c_2\int\psi_3^*\psi_2\dd x+c_3^*c_3\int\psi_3^*\psi_3\dd x+...
\end{align*}
Now if we add vertically down the columns, we obtain
\begin{equation*}
    \sum_{m=1}^\infty c_m^* c_1 \int \psi_m^*\psi_1 \dd x + \sum_{m=1}^\infty c_m^* c_2 \int \psi_m^*\psi_2 \dd x + \sum_{m=1}^\infty c_m^* c_3 \int \psi_m^*\psi_3 \dd x +...
\end{equation*}
which is equivalent to
\begin{equation*}
    \sum_{n=1}^\infty\sum_{m=1}^\infty c_m^*c_n\int\psi_m^*\psi_n\dd x = \sum_{n=1}^\infty\sum_{m=1}^\infty c_m^*c_n\delta_{mn}= \sum_{n=1}^\infty\abs{c_n}^2
\end{equation*}
as promised.
\section{The Harmonic Oscillator}
\subsection{algebraic method}
Here is a short proof of Equation [55] in the book:
\begin{equation*}
    [a_-,a_+] = (a_-a_+)-(a_+a_-) = \frac{1}{\hbar\omega}H+\frac{1}{2} - \frac{1}{\hbar\omega}-\frac{1}{2} = 1 \qed
\end{equation*}
\hrule \vspace{5mm}\noindent
Attached next page is a pdf file of page 43 with some arrows to draw connections.
\includepdf[]{pdf_2_3_1_1}

\noindent
Equation [65] in the book states that
\begin{equation*}
    \boxed{ a_+a_-\psi_n = n\psi_n,~~a_-a_+\psi_n = (n+1)\psi_n }
\end{equation*}
Here is a proof. From Equations [57] and [61] in the book, we have
\begin{align*}
\begin{dcases}
\hbar\omega(a_\pm a_\mp\pm\frac{1}{2})\psi_n = E_n\psi_n \\
\psi_n = A_n(a_+)^n\psi_0;~~E_n = \qty(n+\frac{1}{2})\hbar\omega    
\end{dcases}
\end{align*}
Substituting the expression for $E_n$ into the first case above, we arrive at
\begin{align*}
    \begin{dcases}
    \hbar\omega\qty(a_+a_-+\frac{1}{2}\psi_n) = \qty(n+\frac{1}{2})\hbar\omega\psi_n\\
    \hbar\omega\qty(a_-a_+-\frac{1}{2}\psi_n) = \qty(n+\frac{1}{2})\hbar\omega\psi_n
    \end{dcases}
\end{align*}
cancel out $\hbar\omega$ and move the $1/2 \psi_n$ terms to the RHS to obtain
\begin{align*}
    \begin{dcases}
    a_+a_-\psi_n = n\psi_n \\
    a_-a_+\psi_n = (n+1)\psi_n\qed
    \end{dcases}
\end{align*}



\subsection{analytic method}
As stated at the beginning of this subsection, the Schrödinger equation is equivalent to 
\begin{equation*}
    \boxed{\dv[2]{\psi}{\xi} = (\xi^2-K)\psi}
\end{equation*}
where $K \equiv 2E/\hbar\omega$, and here is a proof. First note that 
\begin{equation*}
    \xi\equiv\sqrt{\frac{m\omega}{\hbar}}x \implies\dv[]{\xi}{x}=\sqrt{\frac{m\omega}{\hbar}}
\end{equation*}
The Schrödinger equation (with harmonic oscillator potential) says that
\begin{equation*}
    -\frac{\hbar^2}{2m}\dv[2]{\psi}{x}+\frac{1}{2}m\omega^2x^2\psi=E\psi
\end{equation*}
Now rewrite the second derivative with respect to $x$ as
\begin{equation*}
    \dv{x}\qty(\dv[]{\psi}{x}) = \dv{\xi}\dv{\xi}{x}\qty(\dv{\xi}\dv{\xi}{x}\psi) =\sqrt{\frac{m\omega}{\hbar}}\dv{\xi}\qty(\sqrt{\frac{m\omega}{\hbar}}\dv{\psi}{\xi}) = \frac{m\omega}{\hbar}\dv[2]{\psi}{\xi}
\end{equation*}
so the S.E. becomes
\begin{equation*}
    -\frac{\hbar^2}{2m}\frac{m\omega}{\hbar}\dv[2]{\psi}{\xi} + \frac{1}{2}m\omega^2x^2\psi = E\psi \iff -\frac{\hbar^2}{2m}\frac{m\omega}{\hbar}\dv[2]{\psi}{\xi}+\frac{1}{2}m\omega^2\frac{\hbar}{m\omega}\xi^2\psi=E\psi
\end{equation*}
cancel things out and multiply both sides by $-2/\hbar\omega$,
\begin{equation*}
    -\frac{2}{\hbar\omega}\times\qty(\frac{-\hbar\omega}{2}\dv[2]{\psi}{\xi}+\frac{1}{2}\hbar\omega\xi^2\psi=E\psi)
\end{equation*}
Therefore,
\begin{equation*}
    \dv[2]{\psi}{\xi} = \qty(\xi^2-\frac{2E}{\hbar\omega})\psi = (\xi^2-K)\psi \qed
\end{equation*}
\nl
This subsection introduces the \textbf{recursion formula} (which is entirely equivalent to the Scrhödinger equation):
\begin{equation*}
    a_{j+2} = \frac{(2j+1-K)}{(j+1)(j+2)}a_j
\end{equation*}
where $h(\xi) = a_0 + a_1\xi + a_2\xi^2 + a_3\xi^3+...$ From this, we see that
\begin{equation*}
    a_{j+2} \approx\frac{2}{j}a_j
\end{equation*}
for large $j$. Griffiths states here that, as a result, $h(\xi) \propto \exp(\xi^2)$; hence the series $h(\xi)$ must \textit{terminate}. Here is a slightly different explanation: The function $\exp(\xi^2)$ can be expressed as
\begin{equation*}
    \exp(\xi^2) = 1 + \xi^2 + \frac{\xi^4}{2!} + \frac{\xi^6}{3!} + ... = \sum_{j = 0, 2, 4}^\infty b_j \xi^j ~~~,~~~b_j = \frac{1}{(j/2)!}
\end{equation*}
Therefore, for large $j$,
\begin{equation*}
    \frac{b_{j+2}}{b_j} = \frac{(j/2)!}{[(j+2)/2]!} = \frac{1}{(j+2)/2}\approx\frac{2}{j}
\end{equation*}
Now, from above, $a_{j+2} \approx\frac{2}{j}a_j$ for large $j$, so
\begin{equation*}
    \frac{a_{j+2}}{a_j} \approx \frac{2}{j}
\end{equation*}
and we see that $a_j$ goes like $b_j$, so $h(\xi)$ indeed goes like $\exp(\xi^2)$ for large $j$. And if $h(\xi)$ goes like $\exp(\xi^2)$, then $\psi(\xi) = h(\xi)\exp(-\xi^2/2)$ goes like $\exp(\xi^2/2)$ which blows up as $\xi$ goes to infinity (not normalizable). Thus we conclude that $h(\xi)$ must terminate at some point (for if the series is finite then we need not worry about convergence).
\section{}
\lipsum[4-6]
\section{}
\lipsum[13-14]
\section{}
\lipsum[9-13]




\end{document}

